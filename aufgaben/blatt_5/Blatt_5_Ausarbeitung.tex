\documentclass[a4paper,12pt,titlepage]{article}

\usepackage[german,ngerman]{babel}
\usepackage{fontspec}
\setmainfont{Calibri}
\usepackage{graphicx}
\usepackage{hyperref}
\usepackage{caption}

\begin{document}

\begin{titlepage}
    \centering
    \vspace*{2cm}
    {\LARGE\bfseries Automaten und formale Sprachen Blatt 5\par}
    \vspace{2cm}
    {\Large Jan Lucca Agricola (275867) \& Jakob Schulz (275258)\par}
    \vspace{2cm}
    {\large\today\par}
\end{titlepage}

\section{Aufgabe}
$L(G) = \{ab, aba, abaa, abaaa, abaaaa,...\}$
\section{Aufgabe}
$\Sigma = \{a,b\}\\
V = \{A,B, S\}\\
P = \{S \rightarrow Sa, S \rightarrow Ab, A \rightarrow Ba, B \rightarrow \epsilon\}$
\section{Aufgabe}
$L(G) = \{b, ab, aab, aabb, aabbb, aaab, aaabb, aaabbb, aaabbbb, aaaab,...\}$
\section{Aufgabe}
Die Grammatik G ist nicht vom Typ-3, weil ein Nichtterminal zu einem Terminal und zwei Nichtterminalen abgeleitet werden kann $(S \rightarrow Sab)$\\
\\
Äquivalente Typ-3-Grammatik:\\
$G' = (\{S, A, B, C\}, \{a, b\}, P', S)\\
P' = \{S \rightarrow Ab, S \rightarrow Ba, A \rightarrow Sa, B \rightarrow Ca, C \rightarrow \epsilon\}
$
\section{Aufgabe}
$G = (\{S, A, B\}, \{a, b\}, P, S)\\
P = \{\\S \rightarrow Ab,\\ S \rightarrow Aa,\\ A \rightarrow Bb,\\ A \rightarrow Ba,\\ B \rightarrow Sb,\\ B \rightarrow Sa,\\ S \rightarrow \epsilon\\\}
$
\section{Aufgabe}
\subsection{Blatt 3 Aufgabe 2}
$G = (\{S, A, B, C, D\}, \{a, b\}, P, S)\\
P = \{\\
S \rightarrow Aa\\
A \rightarrow Ba,\\
A \rightarrow Bb,\\
B \rightarrow Ca,\\
B \rightarrow Da,\\
B \rightarrow Db,\\
D \rightarrow Aa,\\
D \rightarrow Ab,\\
C \rightarrow \epsilon\\
\}$
\subsection{Blatt 3 Aufgabe 3}
$G = (\{S, A, B, C\}, \{a, b\}, P, S)\\
P = \{\\
S \rightarrow Aa,\\
A \rightarrow Bb,\\
B \rightarrow Ca,\\ 
B \rightarrow Sa,\\
C \rightarrow \epsilon\\
\}$
\section{Aufgabe}
$G = (\{S, A, B, C\}, \{a, b\}, P, S)\\
P = \{\\
S \rightarrow S(S), \\
S \rightarrow \epsilon\\
\}
\end{document}
